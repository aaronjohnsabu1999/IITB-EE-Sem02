
\documentclass[12pt]{report}

\usepackage{amssymb}
\usepackage{amscd,enumerate,amsfonts,calc,amsmath,verbatim,xypic}

\pagestyle{plain}
\textwidth1710pt
\topmargin -.5in
\textwidth 6in
\textheight 9in
\hoffset=-0.7in
\textheight600pt
\voffset=.5in

\newcommand{\mb}{\mbox}
\newcommand{\eop}{\hfill{\rule{2mm}{2mm}}}
\newcommand{\bd}{\boldmath}
\newcommand{\dm}[1]{{ \displaystyle{#1}}}
\newcommand{\vone}{\vskip 2ex}
\newcommand{\vhalf}{\vskip .5ex}
\newcommand{\vonehalf}{\vskip 3ex}
\newcommand{\vtwo}{\vskip 4ex}
\newcommand{\vthree}{\vskip 6ex}
\newcommand{\hhalf}{\mbox{\hspace{0.5em}}}
\newcommand{\hone}{\mbox{\hspace{1em}}}
\newcommand{\honehalf}{\mbox{\hspace{1.5em}}}
\newcommand{\htwo}{\mbox{\hspace{2em}}}
\newcommand{\hthree}{\mbox{\hspace{3em}}}
\newcommand{\hsix}{\mbox{\hspace{6em}}}
\newcommand{\height}{\mbox{\hspace{8em}}}
\newcommand{\hnine}{\mbox{\hspace{9em}}}

\newtheorem{chunk}{}
\newtheorem{num}{}
\newtheorem{thm}[num]{{\bf Theorem}}
\newtheorem{lemma}[num]{{\bf Lemma}}
\newtheorem{rmk}{{\bf Remark}}
\newtheorem{prop}[num]{{\bf Proposition}}
\newtheorem{cor}[num]{{\bf Corollary}}
\newtheorem{defn}{{\bf Definition}}
\newtheorem{ex}{{\bf Example}}

\newcommand{\ses}[5]{\ensuremath
{0\ \rar\ #1\ \overset{#4}\rar\ #2\ \overset{#5}\rar\ #3\ \rar\ 0\ }}

\newcommand{\anvec}[3]{\ensuremath{#1_#2, \ldots, #1_#3 }}

\newcommand{\C}{\mathbb C}
\newcommand{\F}{\mathbb F}
\newcommand{\N}{\mathbb N}
\newcommand{\R}{\mathbb R}
\newcommand{\Z}{\mathbb Z}
\newcommand{\xs}[2]{x^{s_{#2}}_{#1}}
\newcommand{\hs}[1]{h^{s_{#1}}}
\newcommand{\ts}{\tilde{\sigma}}
\newcommand{\bs}{\bar{\sigma}}
\newcommand{\p}{{\sf p }}
\newcommand{\sk}{{\sf k }}
\newcommand{\sL}{{\sf L }}
\newcommand{\mQ}{{\mathfrak q }}
\newcommand{\mP}{{\mathfrak p }}
\newcommand{\m}{{\mathfrak m }}
\newcommand{\sn}{{\sf n }}
\newcommand{\e}{\;=\;}


\def\X{{\underline X}}
\def\Y{{\underline Y}}
\def\lm{{\lambda}}
\def\hgt{\text{ht}}
\def\RB{{\overline{\RR}}}
\def\1{{1\hskip-0.25em{\rm 1}}}
\def\Rn{\RR^n}
\def\Rnn{\RR^{n\times n}}
\def\Rm{\RR^m}
\def\RN{\RR^N}
\def\Rmm{\RR^{m\times m}}
\def\Rnm{\RR^{n\times m}}
\def\Rmn{\RR^{m\times n}}
\def\ZZ{{ Z\!\!\! Z}}
\def\CC{{\ \rlap{\raise 0.4ex 
\hbox{$\scriptscriptstyle |$}}\hskip -0.2em C}}
\def\Cn{\CC^n}
\def\Cnn{\CC^{n\times n}}
\def\Cm{\CC^m}
\def\CN{\CC^N}
\def\mm{{\frak m}}

\def\rar{{\rightarrow}}
\def\lrar{{\longrightarrow}}
\def\Lrar{{\Longrightarrow}}
\def\wrt{{with respect to }}
\def\sop{{system of parameters }}
\def\fg{{finitely generated }}
\def\nr{{Noetherian }}
\def\CM{{Cohen-Macaulay }}
\def\rr{{ Ratliff-Rush}}
\def\Wlg{{Without loss of generality }}
\def\wlg{{without loss of generality }}
\def\wmat{{we may assume that }}
\def\nzd{{non-zerodivisor }}
\def\tfae{{the following are equivalent}}
\def\Tfae{{The following are equivalent}}
\def\dim{\text{dim}}
\def\depth{\text{depth}}
\def\spec{\text{Spec}}
\def\Ass{\text{Ass}}
\def\det{\text{det}}
\def\ka{\kappa}

\begin{document}

\centerline{\sc A Theorem and an Application}
\centerline{\rm A Theorem and an Application}
\centerline{\it A Theorem and an Application}
\centerline{\sf A Theorem and an Application}
\centerline{\bf \tt A Theorem and an Application}
\centerline { ${\mathfrak {A \ Theorem \ and \ an \ Application}}$} 
\centerline {\Large {A Theorem and an Application}} 
\centerline {\small {A Theorem and an Application}} 

\begin{thm}\label{T1}
$1 = 2$.
\end{thm}

\noindent
{\bf Proof:} Let $a = b$. Then $a^2 = ab = b^2$ which gives us $a^2 - ab = a^2 - b^2$. Dividing both sides by $a - b$, we get $a = a + b$, i.e. $a = a + a$. Thus starting with a non-zero $a$, we see that $1 = 2$. \hfill{$\square$}

\begin{cor}
$0 = 1$.
\end{cor}
{\bf Proof:} By Theorem \ref{T1}, $1 = 2$. Subtracting $1$ from both sides, we get the corollary.\hfill{$\square$}\\

\centerline{\bf An Application}
Fields do not exist, since we $0 \neq 1$ in a field.\\ 

\pagebreak

\noindent
{\bf Commutative Diagrams, Tables, Equations and Arrays:}
\begin{chunk}
\[
\xymatrixrowsep{1.8pc} \xymatrixcolsep{2.2pc}
\xymatrix{
0\ar@{->}[r]&
N\ar@{->}[r]^{\varphi}\ar@{->}[d]^{\alpha}&R^2\ar@{->}[d]^{\theta}\ar@{->}[r]^{\psi}&M\ar@{->}[d]_{\beta}\ar@{->}[d]\ar@{->}[r]&0\\
0\ar@{->}[r]&
M^*\ar@{->}[r]^{\psi^*}&(R^2)^*\ar@{->}[r]^{\varphi^*}&N^*&{}
}
\]
\end{chunk}

$$\xymatrix{\mathcal K \ar[r]^{\iota} \ar[rd]^\phi & \mathcal
K[X] \ar[d]^{\pi}\\ & \mathcal K[X]/I}$$



\begin{equation}\label{E:1}
x_{\alpha}(t) = \sum_{n = 0}^{\infty} 
\frac{t^n ({\rm ad}\: X_{\alpha})^n}{n!}.
\end{equation}

$$
\bar{\sigma}((x_{ij})) = f^{-1}(\sigma(x_{ji}))^{-1} f, \ \ 
{\rm where} \ \ 
f =\left( \begin{array}{ccc} 
\ & \ & 1 \\
\ & \ddots & \ \\
1 & \ & \ 
\end{array} \right), \\
$$
$$
f=\left( \begin{array}{ccc} 
a & b & c \\
d & e & f \\
g & h & i 
\end{array} \right), 
$$ 
 
$^{6}\!D_4$ 
$\displaystyle < \left[\frac{n - 1}{2} \right]$ 
\[ g(z) = \left\{ \begin{array}
{r@{\quad \quad}l}
\overline{f(\bar{z})} & z \in G_- \\ f(z) & z \in G_+ \cup G_0
\end{array} \right. \]  \\

\begin{tabular}{|r||r@{--}l|p{1.25in}|}
\hline
\multicolumn{4}{|c|}{Book Store}\\ \hline \hline
&\multicolumn{2}{c|}{Price}&\\ \cline{2-3}
\multicolumn{1}{|c||}{Year}
& \multicolumn{1}{r@{\,\vline\,}}{low}& high & \multicolumn{1}{c|}{Comments}\\ \hline
1971 & 97 & 245 & Not Bad.\\ \hline
72 & 245 & 2001  & Not so good this year.\\ \hline 
\end{tabular}

\begin{equation}\label{E:1.5}
if \ a \ normal \ subgroup \  
N \subset {\bf G}(K) \ is \ {\mathcal A}-adically \ open, 
\end{equation}

\vfill{\eject}

\noindent
{\LARGE Symbols:}\\
$$\psi \quad \chi \quad \xi \quad \l \quad \mp \quad \iff \quad \div \quad \ss \quad \S$$
$n \geq 5,$ $n \geqslant 5,$ \\

\bigskip

\noindent
{\Large Self-defined Macros:}\\

\noindent
Consider the vectors \anvec{X}{1}{n} and \anvec{f}{i}{j}.\\

\noindent
We get a short exact sequence \ses{K}{M}{N}{f}{g}\\


\noindent
{\bf \tiny Other Stuff:}

The Page Number is: {\rm \thepage}

$$
z \mapsto \frac{\alpha z + \beta}{\gamma z + \delta}, \ \ \ 
\alpha\delta - \beta\gamma \neq 0, 
$$

 
${\mathbb P}^1(F_p),$
$\pi \colon \tilde{\bf G} \to {\bf G}$ 
\footnote{We will use bold face to denote algebraic groups.}\\

$\mid F(K) \mid \quad \quad \quad \tilde{G}$\\
\indent
in Dickson's own words \cite{D}, ....\\
$\delta({\rm diag}(1, \ldots , 1, d)) = d [D^* , D^*]$ 
$SL_m^+(D);$\\

\hfill{\parbox[t]{7cm}{\rm It is .... }}

\vskip3mm

\hskip180pt E.~Cartan (1936)\footnote{Quoted after L.~Solomon's 
review in MR of \cite{Ca}}

\bigskip 
$$
{\mathfrak g} = {\mathfrak h} \oplus (\oplus_{\alpha \in R} 
{\mathfrak g}_{\alpha}){\mathcal L}$$
 
the sum (\ref{E:1})
${\rm End}\: {\mathfrak g} \otimes_{\mathbb C} {\mathbb C}[[t]]$
But as we pointed out in \S 1, 
$\bar{\sigma}$ 
$\bar{\ }$ 
$^2\!A_n$
 

\medskip

\S\S 2.2B-C. 
$\mid K \mid \geqslant 4,$ 

$^{3,6}\!D_4$ and $^2\!E_6.$ 

\theequation\\

\setcounter{equation}{10}

current equation number is : \theequation

$$
\Gamma_l = \{ X \in GL_m({\mathbb Z}_p) \mid X \equiv E_3 ({\rm mod}\: 
p^l) \}
$$

$\Delta({\mathcal G})$ 
$\displaystyle (3\frac{1}{2})$,
$\displaystyle \left( 3\frac{1}{2} \right)$,
$(3\frac{1}{2})$


\smallskip 

\bibliographystyle{amsplain}
\begin{thebibliography}{100}


\bibitem[C]{Ca} R.W.~Carter, {\it Simple groups of Lie type,} 
John Wiley \& Sons, London-New York-Sydney, 1972. 

\bibitem[5]{D} L.E.~Dickson, {\it Linear Groups,} Dover Publications, New 
York, 1958; 1$^{\rm st}$ edition -- 1901. 

\end{thebibliography} 


  

\end{document}










